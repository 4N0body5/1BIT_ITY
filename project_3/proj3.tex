
\documentclass[11pt]{article}
\usepackage[a4paper, total={17cm,24cm},left=2cm,top=3cm]{geometry}
\usepackage{times}
\usepackage[czech]{babel}
\usepackage[utf8]{inputenc}

\usepackage{pdflscape}
\usepackage{array}
\usepackage{multirow}
\usepackage[czech,ruled,noline,linesnumbered]{algorithm2e}

\usepackage[unicode]{hyperref}
            %so the reference really does work
            
\urlstyle{same}

\begin{document}

\begin{titlepage}
    \begin{center}
        {\Huge\scshape Vysoké učení technické v~Brně \\
        \huge\vspace{0.4em}Fakulta informačních technologií}
        \vspace{\stretch{0.382}}
        
        {\LARGE Typografie a~publikování\,--\,3.~projekt \\
        \Huge\vspace{0.3em}Tabulky a obrázky}
        \vspace{\stretch{0.618}}
    \end{center}
    
    {\Large \today \hfill Natália Bubáková}
\end{titlepage}


%%%    page 1    %%%


\section{Úvodní strana}

Název práce umístěte do zlatého řezu a~nezapomeňte uvést dnešní datum a~vaše jméno a~příjmení

\section{Tabulky}
Pro sázení tabulek můžeme použít buď prostředí\texttt{ tabbing }nebo prostředí\texttt{ tabular}.

\subsection{Prostředí\texttt{ tabbing}}
Při použití\texttt{ tabbing }vypadá tabulka následovně:

\begin{tabbing}
  Vodní melouny \quad \= Cena \quad      \= Množství  \quad  \kill
  \bfseries Ovoce     \> \bfseries Cena  \> \bfseries Množství \\
  Jablka              \> 25,90           \> 3\,kg     \\
  Hrušky              \> 27,40           \> 2,5\,kg   \\
  Vodní melouny       \> 35,-            \> 1\,kus    \\
\end{tabbing}

\noindent Toto prostředí se dá také použít pro sázení algoritmů, ovšem vhodnější je použít prostředí\texttt{ algorithm }nebo\texttt{ algorithm2e }(viz sekce \ref{sek3}).

\subsection{Prostředí\texttt{ tabular}}

Další možností, jak vytvořit tabulku, je použít prostředí\texttt{ tabular}. Tabulky pak budou vypadat takto\footnote{Kdyby byl problem s\texttt{ cline,} zkuste se podívat třeba sem: \url{http://www.abclinuxu.cz/tex/poradna/show/325037.}}:
\bigskip

\begin{table}[h]
\centering
\catcode`\-=12
\begin{tabular}{|l|c|c|}
    \hline
                    &   \multicolumn{2}{c|}{\bfseries Cena} \\ \cline{2-3} 
    \bfseries Měna  &  \bfseries nákup  & \bfseries prodej  \\ \hline
        EUR         &   25,227          &   26,943          \\ \hline
        GBP         &   29,368          &   31,492          \\ \hline
        USD         &   21,260          &   21,661          \\ \hline
\end{tabular}
\caption{Tabulka kurzů k~dnešnímu dni}
\label{tab1}
\end{table}

\bigskip

\begin{table}[h] 
\centering 
\catcode`\-=12
\begin{tabular}{|>{\bfseries}c|c|}
    \hline
    $A$ & ${\neg}A$ \\ \hline
     P  &  N        \\ \hline
     O  &  O        \\ \hline
     X  &  X        \\ \hline
     N  &  P        \\ \hline
\end{tabular}
\begin{tabular}{|c|>{\bfseries}c|c|c|c|c|}
    \hline 
    \multicolumn{2}{|c|}{\multirow{2}{*}{$A \land B$}}
     & \multicolumn{4}{|c|}{$B$}                \\ \cline{3-6}
    \multicolumn{2}{|c|}{} & \bfseries P 
     & \bfseries O & \bfseries X & \bfseries N  \\ \hline 
    \multirow{4}{*}{$A$}
     &  P  &  P  &  O  &  X  &  N   \\ \cline{2-6} 
     &  O  &  O  &  O  &  N  &  N   \\ \cline{2-6}
     &  X  &  X  &  N  &  X  &  N   \\ \cline{2-6}
     &  N  &  N  &  N  &  N  &  N   \\ \hline 
\end{tabular}
\begin{tabular}{|c|>{\bfseries}c|c|c|c|c|}
    \hline 
    \multicolumn{2}{|c|}{\multirow{2}{*}{$A \lor B$}}
     &\multicolumn{4}{|c|}{$B$}                 \\ \cline{3-6}
    \multicolumn{2}{|c|}{} & \bfseries P 
     & \bfseries O & \bfseries X & \bfseries N  \\ \hline
    \multirow{4}{*}{$A$}
     &  P  &  P  &  P  &  P  &  P   \\ \cline{2-6} 
     &  O  &  P  &  O  &  P  &  O   \\ \cline{2-6}
     &  X  &  P  &  P  &  X  &  X   \\ \cline{2-6}
     &  N  &  P  &  P  &  P  &  P   \\ \hline 
\end{tabular}
\begin{tabular}{|c|>{\bfseries}c|c|c|c|c|}
    \hline 
    \multicolumn{2}{|c|}{\multirow{2}{*}{$A \rightarrow B$}}
     &\multicolumn{4}{|c|}{$B$}                 \\ \cline{3-6}
    \multicolumn{2}{|c|}{} & \bfseries P 
     & \bfseries O & \bfseries X & \bfseries N  \\ \hline
    \multirow{4}{*}{$A$}
     &  P  &  P  &  O  &  X  &  N   \\ \cline{2-6} 
     &  O  &  P  &  O  &  P  &  O   \\ \cline{2-6}
     &  X  &  P  &  P  &  X  &  X   \\ \cline{2-6}
     &  N  &  P  &  P  &  P  &  P   \\ \hline
\end{tabular}
\caption{Protože Kleeneho trojhodnotová logika už je \uv{zastaralá}, uvádíme si zde příklad čtyřhodnotové logiky}
\label{tab2}
\end{table}



%%%    page 2    %%%


\section{Algoritmy} \label{sek3}

Pokud budeme chtít vysázet algoritmus, můžeme použít prostředí\texttt{ algorithm}\footnote{Pro nápovědu, jak zacházet s prostředím\texttt{ algorithm,} můžeme zkusit tuhle stránku: \\
\url{http://ftp.cstug.cz/pub/tex/CTAN/macros/latex/contrib/algorithms/algorithms.pdf}.} nebo \texttt{algorithm2e}\footnote{Pro \texttt{algorithm2e} zase tuhle: \url{http://ftp.cstug.cz/pub/tex/CTAN/macros/latex/contrib/algorithm2e/doc/algorithm2e.pdf}.}.\\
Příklad použití prostředí\texttt{ algorithm2e }viz Algoritmus \ref{alg1}.

\vspace{2em}

\begin{algorithm}[H]
    \label{alg1}
    \caption{\textsc{Fast}\uppercase{slam}}
    \SetNlSty{}{}{:}
    \SetKwFor{For}{for}{do}{end\ for}
    \SetInd{1em}{1em}
    \SetNlSkip{-1.4em}
    
    \KwIn{$(X_{t-1},u_{t},z_{t})$}
    \KwOut{$X_{t}$}
    \Indp \Indp
    \BlankLine
    
    $\overline{X_{t}}=X_{t}=0$ \\
    \For{ $k=1$ to $M$ }
     {
        $x_{t}^{[k]} = sample\_motion\_model(u_{t}, x_{t-1}^{[k]})$ \\
        $\omega^{[k]}_{t} = measurment\_model(z_{t}, x_{t}^{[k]}, m_{t-1})$ \\
        $m^{[k]}_{t} = updated\_occupancy\_grid(z_{t}, x_{t}^{[k]}, m_{t-1}^{[k]})$ \\
        $\overline{X_{t}} = \overline{X_{t}} + \langle x_{x}^{[m]}, \omega^{[m]}_{t}\rangle$ \\
     }
    \For{ $k=1$ to $M$ }
     { 
        draw $i$ with probability $\approx \omega^{[i]}_{t}$   \\
        add $\langle x^{[k]}_{x},m^{[k]}_{t}\rangle$ to $X_{t}$
     }
    \Return{ $X_{t}$ }
\end{algorithm}

\vspace{1.5em}


\section{Obrázky}
Do našich článků můžeme samozřejmě vkládat obrázky. Pokud je obrázkem fotografie,můžeme klidně použít bitmapový soubor. Pokud by to ale mělo být nějaké schéma neboněco podobného, je dobrým zvykem takovýto obrázek vytvořit vektorově.

\begin{figure}[h]
    \centering
    \scalebox{0.4}{ \includegraphics{etiopan.eps}
     \reflectbox{\includegraphics{etiopan.eps}} }
    \caption{Malý Etiopánek a~jeho bratříček}
    \label{fig1}
\end{figure}
\bigskip

\pagebreak

%%%    page 3    %%%

Rozdíl mezi vektorovým\,\dots

\begin{figure}[h]
    \centering
    \scalebox{0.4}{ \includegraphics{oniisan.eps}}
    \caption{Vektorový obrázek}
    \label{fig2}
\end{figure}
\bigskip
\noindent\dots\,a bitmapovým obrázkem

\begin{figure}[h]
    \centering
    \scalebox{0.6}{ \includegraphics{oniisan2.eps}}
    \caption{Bitmapový obrázek}
    \label{fig3}
\end{figure}
\bigskip

\noindent se projeví například při zvětšení.\par
Odkazy (nejen ty) na obrázky \ref{fig1},\ref{fig2} a~\ref{fig3}, na  
tabulky \ref{tab1} a~\ref{tab2} a~také na algoritmus \ref{alg1} jsou udělány pomocí křížových odkazů. Pak je ovšem potřeba zdrojový soubor přeložit dvakrát.\par
Vektorové obrázky lze vytvořit i přímo v~\LaTeX{}u, například pomocí prostředí\texttt{ picture}.


%%%    page 4    %%%

\newpage
\begin{landscape}

    \begin{figure}
    \centering
    \setlength{\unitlength}{1mm}
    \begin{picture}(200,100)
        
        \linethickness{1pt}
        \put(0,0){\framebox(200,100){}}
        
        \multiput(100,10)(0,20){2}{\line(2,1){30}}
        \multiput(30,10)(0,20){2}{\line(1,0){70}}
        \multiput(30,10)(70,0){2}{\line(0,1){20}}
        \put(130,25){\line(0,1){20}}
        \put(30,30){\line(3,1){18}}
        %\put(40,33.5){\line(1,0){30}}
        \multiput(32,30)(16.7,0){5}{\line(0,1){5}}
        \multiput(106.8,34)(7.4,3.8){4}{\line(0,1){4}}
        \multiput(32,31.5)(0,1){4}{\line(1,0){67}}
        \multiput(98.8,31.5)(0,0.9){4}{\line(2,1){30}}
        \multiput(32,31.5)(0,0.9){4}{\line(3,1){15}}
        \multiput(47,36)(3,0){2}{\line(0,1){20}}
        \put(47,36){\line(1,0){3}}
        \put(73,36){\line(1,0){3}}
        \multiput(73,36)(3,0){2}{\line(0,1){20}}
        \multiput(85,42)(2.5,1){2}{\line(0,1){17}}
        \multiput(45,56)(0,2){2}{\line(1,0){32}}
        \multiput(45,56)(32,0){2}{\line(0,1){2}}
        \multiput(76.5,56)(0,2){2}{\line(3,1){15}}
        \put(45,58){\line(4,1){20}}
        \multiput(65,63)(26.5,0){2}{\line(0,1){20}}
        \multiput(65,63)(0,20){2}{\line(1,0){26.5}}
        \multiput(91.5,63)(0,-20){2}{\line(1,0){23}}
        \multiput(91.5,43)(23,0){2}{\line(0,1){20}}
        \put(91.5,63.3){\line(3,1){10}}
        \put(114.5,63.3){\line(5,2){8}}
        \put(91.5,83.3){\line(4,1){10}}
        \put(65,83.3){\line(4,1){11}}
        \put(75,86){\line(1,0){26.5}}
        \put(101.5,66.4){\line(0,1){20}}
        \put(101.5,66.4){\line(1,0){21}}
        \put(122.5,47.5){\line(0,1){18.5}}
        \put(113.5,42.8){\line(2,1){9.2}}
        \put(122.5,49.5){\line(1,0){6.5}}
        \put(50,38){\line(3,1){14.5}}
        \put(52,37){\line(3,1){15}}
        \put(76,38.5){\line(2,1){7}}
        \put(52,37){\line(1,0){21}}
        \put(67,41.8){\line(1,0){6}}
        \put(76,41.8){\line(1,0){7}}
        \put(65,43){\line(1,0){7.5}}
        \put(65,43){\line(0,1){12.8}}
        \put(76.5,43){\line(1,0){8}}
        \put(88.5,43){\line(1,0){3}}
        \put(84.5,41.8){\line(3,1){4}}
        \multiput(72,45)(0,10){2}{\line(1,0){1}}
        \multiput(75.8,45)(0,10){2}{\line(1,0){5.1}}
        \multiput(72,45)(9,0){2}{\line(0,1){10}}
        \multiput(95,45)(7.7,0){3}{\line(0,1){10}}
        \multiput(95,45)(0,10){2}{\line(1,0){15.4}}
        \multiput(69,66)(6,0){4}{\line(0,1){10}}
        \multiput(69,66)(0,10){2}{\line(1,0){18}}
        \multiput(116,47)(2.5,1.30){3}{\line(0,1){9}}
        \multiput(116,47)(0,9){2}{\line(2,1){5}}
        \multiput(94,67)(3,1){3}{\line(0,1){9.3}}
        \multiput(94,67)(0,9.3){2}{\line(3,1){5.7}}
        \multiput(103,16)(4.8,2.4){6}{\line(0,1){10}}
        \multiput(103,16)(0,10){2}{\line(2,1){24}}
        \multiput(34,14)(0,11){2}{\line(1,0){30}}
        \multiput(34,14)(10,0){4}{\line(0,1){11}}
        \multiput(66,14)(0,11){2}{\line(1,0){30}}
        \multiput(66,14)(10,0){4}{\line(0,1){11}}
        \multiput(36,0.5)(60,0){2}{\line(0,1){9.5}}
        \put(125,0.5){\line(0,1){22}}
        \put(40,5){\line(1,0){51}}
        \multiput(40,0.5)(8.5,0){7}{\line(0,1){4.5}}
        \multiput(112,0.5)(4.5,2.25){3}{\line(0,1){10}}
        \put(101,5){\line(2,1){20}}
        \put(101.5,0.5){\line(0,1){4.5}}
        \put(107,0.5){\line(0,1){7.5}}
        \put(111.5,0.5){\line(2,1){9.5}}
        \put(165,40){\circle{14.5}}
        \put(170,50){\line(2,3){7}}
        \put(175,45){\line(3,1){23}}
        \put(175,37){\line(6,-1){10}}
        \put(136,47){\line(6,-1){20}}
        \put(141,34){\line(5,1){13}}
        \put(160,50){\line(-1,1){23}}
        \put(10,80){\oval(14.5,14.5)[r]}
        \put(10,73){\qbezier[50](0,0)(7,7)(0,14.4)}
        \put(150,30){\qbezier[200](0,0)(20,3)(50,0)}
        \put(130,28){\qbezier[200](0,2.5)(14,3.5)(40,0)}
        \put(160,16){\qbezier[200](0,4)(13,0)(35,3)}
        \put(130,9){\qbezier[200](0,2)(13,5)(40,0)}
        \put(0.5,25){\qbezier[200](0,2)(10,-1)(29.5,1)}
        \put(9,4){\qbezier[200](0,2)(7,4)(26.5,0)}
        \put(180,80){\qbezier[50](0,1)(4,5)(8,0)}
        \put(172,80){\qbezier[50](0,0)(4,5)(8,1)}
        \put(165,87){\qbezier[50](0,0)(4,7)(6,3)}
        \put(159,87){\qbezier[50](0,3)(4,7)(6,0)}


    \end{picture}
    \caption{Vektorový obrázek moderního bydlení na pobreží, pod krycím jménem \uv{Natalka's little big dream alias Beach Penthouse}}
    \end{figure}
    
\end{landscape}

\end{document}
