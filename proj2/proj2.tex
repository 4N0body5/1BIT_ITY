
\documentclass[twocolumn, 11pt]{article}
\usepackage[a4paper, total={18cm,25cm},left=1.5cm,top=2.5cm]{geometry}
\usepackage{times}
\usepackage[czech]{babel}
\usepackage[IL2]{fontenc}
\usepackage[utf8]{inputenc}

\usepackage{amsmath}
\usepackage{amsthm}
\usepackage{amssymb}

%\usepackage[unicode]{hyperref}
    %so the reference really does work


\newtheorem{definicia}{Definice}
\newtheorem{veta}{Věta}




\begin{document}

\begin{titlepage}
    \begin{center}
        {\Huge\scshape Fakulta informačních technologií \\ \vspace{0.4em}Vysoké učení technické v Brně}
        \vspace{\stretch{0.382}}
        
        {\LARGE Typografie a~publikování\,--\,2. projekt \\ \vspace{0.3em} Sazba dokumentů a~matematických výrazů}
        \vspace{\stretch{0.618}}
    \end{center}
    
    {\Large \the\year \hfill Natália Bubáková (xbubak01)}
\end{titlepage}

\newpage

\section*{Úvod}

V~této úloze si vyzkoušíme sazbu titulní strany, matematic\-kých vzorců, prostředí a~dalších textových struktur obvyklých pro technicky zaměřené texty (například rovnice (\ref{eq1}) nebo Definice \ref{def1} na straně \pageref{def1}).
Rovněž si vyzkoušíme používání odkazů \verb|\ref| a~\verb|\pageref|.\par 
Na titulní straně je využito sázení nadpisu podle optického středu s~využitím zlatého řezu. Tento postup byl probírán na přednášce. Dále je použito odřádkování se zadanou relativní velikostí 0.4\,em a~0.3\,em.\par
V~případě, že budete potřebovat vyjádřit matematickou konstrukci nebo symbol a~nebude se Vám dařit jej nalézt v~samotném \LaTeX{}u, doporučuji prostudovat možnosti ba\-lí\-ku maker \AmS-\LaTeX.

\section{Matematický text}
Nejprve se podíváme na sázení matematických symbolů
a~výrazů v~plynulém textu včetně sazby definic a~vět s~využitím balíku \verb|amsthm|. Rovněž použijeme poznámku pod
čarou s~použitím příkazu \verb|\footnote|. Někdy je vhodné použít konstrukci \verb|\mbox{}|, která říká, že text nemá být zalomen.

\begin{definicia}
    \label{def1}
    \textup{Rozšířený zásobníkový automat} (RZA) je definován jako sedmice tvaru $A=(Q, \Sigma, \Gamma, \delta, q_0, Z_0, F)$, kde:
    \begin{itemize}
        \item[$\bullet$] $Q$ je konečná množina \textup{vnitřních (řídicích) stavů},
        
        \item[$\bullet$] $\Sigma$ je konečná \textup{vstupní abeceda},
        
        \item[$\bullet$] $\Gamma$ je konečná \textup{zásobníková abeceda},
        
        \item[$\bullet$] $\delta$ je \textup{přechodová funkce} $Q \times (\Sigma \cup \{\epsilon\})\times \Gamma^{\ast} \rightarrow 2^{Q \times \Gamma^{\ast}}$,
        
         \item[$\bullet$] $q_0  \in Q$ je \textup{počáteční stav,} $Z_0 \in \Gamma$
         je \textup{startovací symbol zásobníku} a~$F \subseteq Q$
         je množina \textup{koncových stavů.}
         
     \end{itemize}
\end{definicia}
     
Nechť $P=(Q, \Sigma, \Gamma, \delta, q_0, Z_0, F)$ je rozšířený zásobníkový automat. \emph{Konfigurací} nazveme trojici $(q, w, \alpha)\in Q \times \Sigma^{\ast} \times \Gamma^{\ast}$, kde $q$ je aktuální stav vnitřního řízení, $w$ je dosud nezpracovaná část vstupního řetězce a~$\alpha = Z_{i_1} Z_{i_2} \dots Z_{i_k}$
je obsah zásobníku\footnote{$Z_{i_1}$ je vrchol zásobníku}.

 
\subsection{Podsekce obsahující větu a~odkaz}
\begin{definicia} 
    \label{def2}
    \textup{Řetězec $w$ nad abecedou $\Sigma$ je přijat RZA} $A$ jestliže $(q_0, w, Z_0) \overset{*}{\underset{A}{\vdash}} (q_F,\epsilon, \gamma)$ pro nějaké $\gamma \in \Gamma^{\ast}$ a $q_F \in F$. Množinu $L(A) = \{w \mid w \mbox{ je přijat RZA } A\} \subseteq \Sigma^{\ast}$ nazýváme \textup{jazyk přijímaný RZA} $A$.
\end{definicia}


Nyní si vyzkoušíme sazbu vět a~důkazů opět s~použitím balíku \verb|amsthm|.
\begin{veta} 
    Třída jazyků, které jsou přijímány ZA, odpovídá
    \textup{bezkontextovým jazykům}.
\end{veta}

\begin{proof}
    V důkaze vyjdeme z~Definice \ref{def1} a~\ref{def2}.
\end{proof}


\section{Rovnice a odkazy}
Složitější matematické formulace sázíme mimo plynulý
text. Lze umístit několik výrazů na jeden řádek, ale pak je třeba tyto vhodně oddělit, například příkazem \verb|\quad|.

$$
\sqrt[i]{x_i^3}\quad \text{kde } x_i \text{ je } i\text{-té sudé číslo splňující}\quad x_i^{x_i^{i^2}+2} \leq  y_i^{x_i^4}
$$ \par
V rovnici (\ref{eq1}) jsou využity tři typy závorek s různou
explicitně definovanou velikostí.

\begin{eqnarray}
\label{eq1}
x & = & \bigg[  \Big\{ \big[ a+b \big] \ast c \Big\}^d \oplus 2 \bigg]^{3/2} \\
y & = & \lim_{x\to\infty} \frac{\frac{1}{\log_{10} x}}{\sin^2 x + \cos^2 x}\nonumber
\end{eqnarray}

V~této větě vidíme, jak vypadá implicitní vysázení limity $\lim_{n\to\infty} f(n)$ v~normálním odstavci textu. Podobně
je to i s~dalšími symboly jako $\prod_{i=1}^{n} 2^i$ či $\bigcap_{A\in \mathcal{B}} A$.
V~případě vzorců $\lim\limits_{n\to\infty} f(n)$ a~$\prod\limits_{i=1}^n 2^i$ jsme si vynutili méně úspornou sazbu příkazem \verb|\limits|.

\begin{eqnarray}
\int_{b}^{a} g(x)\,\mathrm{d}x & = & -\int\limits_{a}^{b} f(x)\,\mathrm{d}x
\end{eqnarray}

\section{Matice}
Pro sázení matic se velmi často používá prostředí \verb|array| a~závorky (\verb|\left|, \verb|\right|).

$$
 \left( \begin{array}{ccc}
a-b & \widehat{\xi + \omega} & \pi \\
\vec{\mathbf{a}} & \overleftrightarrow{AC} & \hat{\beta} \end{array} \right) = 1 \Longleftrightarrow \mathcal{Q} = \mathbb{R}
$$

$$
\mathbf{A} = \left\|\begin{array}{cccc}
a_{11} & a_{12} & \cdots & a_{1n} \\
a_{21} & a_{22} & \cdots & a_{2n} \\
\vdots & \vdots & \ddots & \vdots \\
a_{31} & a_{32} & \cdots & a_{3n} \end{array} \right\| = \left| \begin{array}{cc} t & u \\ v & w \\ \end{array} \right|= tw-uv
$$ 

Prostředí \verb|array| lze úspěšně využít i jinde.

 $$\binom{n}{k} = \left\{ \begin{array}{cl}
         0 & \mbox{pro }k\,<\,0 \mbox{ nebo }k\,>\,n\\
        \frac{n!}{k!(n-k)!} & \mbox{pro }  0\, \leq \,k\, \leq\, n.\end{array} \right.$$


\end{document}
