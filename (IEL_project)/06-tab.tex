\section{Zhrnutie výsledkov}
\vspace{2cm}
\begin{center}
    \begin{tabular}{|c|c|c|} \hline 
        \textbf{Príklad} & \textbf{Skupina} & \textbf{Výsledky} \\ \hline
        1 & \prvniSkupina & $U_{R6} = 71,4183V$ \qquad \qquad $I_{R6} = 0,0821A$ \\ \hline
        2 & \druhySkupina & $U_{R3} = 70,0309V$ \qquad \qquad $I_{R3} = 0,1207A$ \\ \hline
        3 & \tretiSkupina & $U_{R2} = 57,4031V$ \qquad \qquad $I_{R2} = 1,1715A$\\ \hline
        4 & \ctvrtySkupina & $|U_{L_{2}}| = $ \qquad \qquad $\varphi_{L_{2}} = $ \\ \hline
        5 & \patySkupina & $ i_L(t)=\frac{9}{20}+\frac{91}{20}e^{-\frac{4}{5}t}$ \\ \hline
    \end{tabular}
\end{center}
